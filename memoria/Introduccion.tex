\chapter{Introducción}
\label{cap:Introduccion}

%   Definición del problema principal
Los desastres climáticos conforman una realidad sumamente atroz, la cual está presente en los medios de comunicación a día de hoy. Sin ir más lejos, en este 2019 se han organizando numerosas manifestaciones en todo el globo en forma de protesta contra la pasividad ante el cambio climático y concienciar a la población antes de llegar al 2030, el año de no retorno \cite{elpais01}.
% Recurso: Fotos Marchas 15: https://elpais.com/Comentario/1552688956-d7e20feb8fbbb828976c52707f1ea897?gla=es

Si no es uno de los más importantes, el calentamiento global acarreará una de las más catastróficas consecuencias, aunque hay muchas más: la falta de agua potable. Ésta marcará la supervivencia del ser humano, pues no sólo escaseará en el consumo directo, si no en la producción de bienes agrícolas y ganaderos. Acercándonos al campo de la agricultura, sólo en Castilla-La Mancha se emplearon 1.655.033 miles de metros cúbicos de agua en 2016 \cite{ine01}.

%   Soluciones actuales
Las soluciones a la escasez hídrica radican en el empleo de un sistema de riego adecuado al cultivo y tamaño de este. Nos encontramos, de entre todas ellas, dos destacadas: la aspersión y el goteo. Sin embargo, de entre todas estas alternativas, ninguna aporta una solución eficaz en relación al entorno que rodea al terreno, y si lo hacen, sólo pueden aplicarse a zonas de menor tamaño y con mayor control del sus condiciones, como podrían ser invernaderos.

%   Solución obtenida
La tecnología del \emph{Internet de las Cosas}, referido de ahora en adelante como IoT, permite parametrizar situaciones que vivimos diariamente y nos da la posibilidad de actuar frente a posibles cambios en esos datos en un tiempo mínimo. Algunos ejemplos son las ciudades inteligentes o la conectividad entre vehículos.

A lo largo de este proyecto se desarrollará la idea de un sistema IoT de agricultura inteligente, el cual recogerá datos del propio cultivo relativos al crecimiento y cuidado del mismo y posteriormente se analizarán para su debida reacción en el suministro de agua. Con ella, se podrá optimizar el ciclo de riego de la producción con datos que proporciona el mismo y reaccionar ante ellos de forma casi instantánea.

El objetivo principal del proyecto es la reducción de la cantidad de recursos utilizados, ya sean hídricos o eléctricos (consumidos al extraer u obtener los primeros) a la par que se optimiza el ciclo de vida del cultivo proporcionando al cultivo agua únicamente y siempre en el momento en el que lo necesite. Al agricultor se le facilitará una herramienta donde tendrá centralizados todos los terrenos de los que disponga y podrá visualizar el estado de cada uno en tiempo real y desde cualquier plataforma. 