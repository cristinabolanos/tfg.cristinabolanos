
\newpage
\section{Medios}
\label{sec:means}

\subsection{Medios Hardware}
\label{subsec:hardware}

Los medios hardware elegidos para desarrollar la solución aportada en este proyecto son:

\begin{itemize}
    \item \textit{Open Garden Outdoor Kit} (\cite{open.garden.01}): Se ha elegido este kit para cultivos de exteriores por la sencillez de su diseño y la gran capacidad de resolver nuestras necesidades de datos. Los componentes más relevantes del presente son:
    \begin{itemize}
        \item \textit{Open Garden Shield for Arduino} (pasarela del SubLC, en figura \ref{fig:architecture}).
        \item \textit{Open Garden Node Board} (controlador del nodo recolector en SubLC, en figura \ref{fig:architecture}).
        \item DHT22 de \textit{Open Garden} como sensor de humedad y temperatura.
        \item Sensor de humedad del terreno de \textit{Open Garden}.
        \item Electroválvula de Open Garden, puede verse en \href{https://www.cooking-hacks.com/electro-valve-for-open-garden}{la tienda online de Open Garden}.
    \end{itemize}
    \item Lenovo Z50-70 (\cite{lenovo.01}),  como medio para la programación y configuración de los diferentes dispositivos.
\end{itemize}

\subsection{Medios Software}
\label{subsec:Software}

Los medios software elegidos para desarrollar la solución aportada en este proyecto son:

\begin{itemize}
    \item MQTT (\cite{ibm.mqtt.01}): Empleado para la comunicación entre la pasarela del SubLC con los nodos de los que dispongamos.
    \item IBM Cloud (\cite{ibm.cloud.01}): Proveedor elegido para el alojamiento del subsistema SubI debido a su compromiso de colaboración con la UCLM y el acceso a sus herramientas para sus estudiantes.
    \item Flask (\cite{flask.01}): Utilizado para la creación de aplicaciones web basadas en Python (\cite{python.01}).
    \item SQL (\cite{sql.01}): Lenguaje de bases de datos utilizado para el almacenaje en el entorno \textit{cloud} de aquellos que sean pertinentes.
    \item LaTeX (\cite{latex.01}): Necesario para la realización de la memoria y del presente documento.
    \item Bitbucket (\cite{bitbucket.01}): Utilizado para el seguimiento de versiones del producto y de su documentación. Emplea Mercurial (\cite{mercurial.01}) para su uso.
    
\end{itemize}