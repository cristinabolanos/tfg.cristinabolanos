
\newpage
\section{Métodos y Fases de Trabajo}
\label{sec:methods}

\justifying \setlength{\parindent}{1.27cm}
\normalsize\mdseries

Para la realización de este proyecto se empleará el Proceso Unificado de Desarrollo, en adelante referido como PUD, como metodología de planificación y trabajo. Se ha optado por dicha alternativa por su relativa sencillez, generalidad, y la amplitud de control de software que ofrece.\newline

Para diseñar y realizar los modelos de cualquier sistema, PUD hace uso del \textit{Lenguaje Unificado de Modelado} (\cite{uml.01}), en adelante referido como UML, el cual se ha utilizado en este proyecto en sus diferentes etapas. Cabe mencionar que PUD se caracteriza por estar \textbf{dirigido por casos de uso} y \textbf{centrado en la arquitectura}, mientras se trata de forma \textbf{iterativa e incremental}.\newline

Por lo tanto, y siguiendo la estructura del ciclo de vida de un producto software definido por PUD, se han establecido las siguientes pautas o fases generales a seguir:

\begin{enumerate}
    \item \textbf{Inicio}: Se llevará a cabo una descripción del producto final que se desea conseguir, estudiando el alcance del proyecto, su viabilidad, y la planificación del desarrollo del proyecto. Se obtienen:
    \begin{enumerate}
        \item Modelo de casos de uso, el cual describe las funcionalidades del sistema.
    \end{enumerate}
    \item \textbf{Elaboración}: Se desarrollará y trabajará en el modelo obtenido en la fase anterior profundizando así en la arquitectura del sistema, consiguiendo la línea base de ésta. Se obtienen:
    \begin{enumerate}
        \item Modelo de análisis, con el que desarrollamos las funcionalidades en procesos, interfaces o bancos de datos.
        \item Modelo de diseño, obteniendo un despliegue mucho más especifico del modelo anterior.
    \end{enumerate}
    \item \textbf{Construcción}: Se implementará y probará el producto, tanto su despliegue físico como el software de cada uno de sus componentes. Se obtiene:
    \begin{enumerate}
        \item Modelo de implementación, con el que se conseguirá representar la ejecución total del sistema.
        \item Modelo de pruebas, el cual contiene el conjunto de casos de pruebas unitarias que se realizan al finalizar cada una de las fases del PUD.
    \end{enumerate}
    \item \textbf{Transición}: Al llegar a esta etapa, el producto está en su versión beta y se procede a la evaluación del mismo en busca de errores o deficiencias. Se obtiene:
    \begin{enumerate}
        \item Versión del producto final.
        \item Documentación del desarrollo.
    \end{enumerate}
\end{enumerate}