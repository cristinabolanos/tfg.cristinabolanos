
\newpage
\section{Introducción}
\label{sec:Introduction}

\justifying \setlength{\parindent}{1.27cm}
\normalsize\mdseries

%   Definición del problema
El calentamiento global, la contaminación y la deforestación son conceptos plenamente conocidos y, lamentablemente, constantemente visibles en los medios de actualidad. Todos ellos afectan gravemente a las reservas de agua potable, o de uso en cultivos, del planeta, y por tanto, a la esperanza de vida humana.\newline

En Castilla-La Mancha, nuestra tierra, 7.946.198 hectáreas de terreno son dedicadas al cultivo, y sólo en 474.910 de ellas se despliegan viñedos, sin tener en cuenta las diferentes variedades de uva que los agricultores manchegos pueden o no cultivar (\cite{gob.miapa.01}). Para aprovisionar estos terrenos, se necesita un gran volumen de agua, el cual puede proceder de embalses, aguas depuradas, o acuíferos. Sin embargo, y precedida por los factores perjudiciales para el medio ambiente mencionados al inicio de este documento, la sequía es inminente.\newline

%   Descripciones de soluciones actuales
En la actualidad, el sector agrario emplea distintas técnicas de riego para el suministro de agua. Según el \cite{ine.01}, el agua utilizada para riego en España se despliega en las siguientes opciones:

\begin{figure}[h]
    \centering
    \includegraphics[scale=0.75]{figures/graphics/riego_graphic.png}
    \caption{Sistemas de riego utilizados en España.}
    \label{fig:riego_graphic}
\end{figure}

La mejor opción de entre éstas, en relación con el consumo de agua, es el riego por goteo. Sin embargo, no existe una automatización de esta técnica salvo el uso de contadores para el suministro de agua en ciertos periodos del día. También existen sistemas de riego automáticos que monitorizan la plantación y aprovisionan de agua en forma de aspersión según las condiciones hídricas de la planta, por tiempos, por datos meteorológicos, etc.\newline

%   Objetivos (resumen)
Es por ello que, con el fin de reducir al máximo posible el consumo de agua, con razón de cuidado de plantaciones y cultivos, se propone con este proyecto alcanzar una nueva alternativa a los sistemas de riego convencionales. Para ello, se diseñará e implementará un sistema de control de riego automatizado para porciones de terreno grandes.\newline

%   Descripción de mi solución con figuras etc
La solución optada se compone de dos subsistemas diferenciados:
\begin{enumerate}
    \item Lectores de información y controladores del caudal del agua ({\bfseries SubLC}): Estos componentes tendrán la tarea de monitorizar los distintos factores medibles del entorno de la plantación (temperatura, humedad del suelo, etc.) y controlar el caudal necesitado por el propio terreno en función de estos.
    \item Servicio en la red ({\bfseries SubI}): Se dispondrá un servicio disponible en la red que mostrará los datos recogidos por SubLC al usuario, y además controlará al sistema SubLC.
\end{enumerate}

Se explorará más a fondo la arquitectura del sistema en el capítulo \ref{sec:target} del presente documento.